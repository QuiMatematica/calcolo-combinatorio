\chapter{Esercizi e quiz}
\label{ch:quiz}

%\section{Dall'eserciziario dello stage senior}
%\label{sec:quiz_stage_senior}

\section{Quiz dalle gare distrettuali}
\label{sec:quiz_gare_distrettuali}

\begin{esercizio}[Gare distrettuali 2008]
    \label{ex:distrettuali_2008_1}
    Una banda di ladri vuole aprire la cassaforte di una banca.
    Un basista ha fatto ubriacare il direttore della banca ed è riuscito a sapere che:

    (a) la combinazione è formata da 5 cifre da 0 a 9;

    (b) la combinazione è un numero pari;

    (c) esattamente una delle 5 cifre della combinazione è dispari;

    (d) nella combinazione compaiono quattro cifre diverse, la cifra ripetuta è pari e compare in due posizioni non consecutive.

    Quante sono le combinazioni possibili in base a tali informazioni?

    (A) 3150 \quad (B) 4500 \quad (C) 5400 \quad (D) 7200 \quad (E) 9000.
\end{esercizio}

\begin{esercizio}[Gare distrettuali 2008]
    \label{ex:distrettuali_2008_7}
    In quanti modi si possono ordinare le cifre 1, 2, 4, 7 e 9 affinché formino un numero di cinque cifre divisibile per 11?

    (A) 0 \quad (B) 1 \quad (C) 10 \quad (D) 12 \quad (E) 24.
\end{esercizio}

\begin{esercizio}[Gare distrettuali 2016]
    \label{ex:distrettuali_2016_5}
    Cecilia ha un dado a sei facce (numerate da 1 a 6) e 4 colori a disposizione.
    In quanti modi può colorare le sei facce del dado usando in totale almeno tre colori diversi e facendo in modo che
    facce opposte siano di colori diversi?

    (A) $4^3 \cdot 3^3 \quad$ (B) $3^6 - 2^6 \quad$ (C) $2^6 \cdot 3^2 \quad$ (D) $2^4 \cdot 3 \cdot 5 \cdot 7 \quad$

    (E) Nessuna delle precedenti
\end{esercizio}
%\section{Altri quiz}
%\label{sec:quiz_altri}

