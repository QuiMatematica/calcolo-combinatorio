\chapter{Soluzioni}
\label{ch:soluzioni}

\begin{soluzione}{ex:distrettuali_2008_1}
    Cominciamo con il contare in quanti modi possiamo collocare la cifra dispari: essa può essere in qualunque posizione,
    tranne nella quinta che deve essere necessariamente pari.

    Indicando con $d$ la cifra dispari abbiamo le seguenti situazioni:

    \begin{center}
    \begin{tabular}{|c|c|c|c|c|}
        \toprule
        $d$ & & & & \\
        \midrule
        & $d$ & & & \\
        \midrule
        & & $d$ & & \\
        \midrule
        & & & $d$ & \\
        \bottomrule
    \end{tabular}
    \end{center}

    Ora contiamo in quanti modi possiamo collocare le due cifre pari uguali.
    Esse devono occupare due posizioni delle quattro rimaste libere, inoltre non devono essere consecutive.
    Possiamo dividere il conteggio in due parti: inizialmente contiamo le possibili combinazioni, poi togliamo il numero
    di casi in cui le due cifre sono consecutive.

    La prima parte del conteggio è comune alle quattro posizioni della cifra dispari: dobbiamo scegliere due caselle
    in cui posizionare le cifre uguali.
    Quindi abbiamo:
    \begin{equation*}
        C_{4, 2} = \binom{4}{2} = \dfrac{4!}{2! 2!} = 6
    \end{equation*}

    Considerando tutte e quattro le posizioni della cifra dispari, abbiamo al momento $6 \cdot 4 = 24$ possibilità.

    Però ora dobbiamo togliere il numero di volte in cui le due cifre uguali si trovano in posizione consecutiva:
    \begin{itemize}
        \item nel primo caso dobbiamo togliere 3 possibilità ($2^a$ e $3^a$ cifra, $3^a$ e $4^a$ cifra, $4^a$ e $5^a$ cifra);
        \item nel secondo caso dobbiamo togliere 2 possibilità ($3^a$ e $4^a$ cifra, $4^a$ e $5^a$ cifra);
        \item nel terzo caso dobbiamo togliere 2 possibilità ($1^a$ e $2^a$ cifra, $4^a$ e $5^a$ cifra);
        \item nel quarto caso dobbiamo togliere 2 possibilità ($1^a$ e $2^a$ cifra, $2^a$ e $3^a$ cifra).
    \end{itemize}

    In totale dobbiamo togliere 9 possibilità.
    Quindi tenendo conto della cifra dispari e delle due cifre pari uguali ma non consecutive abbiamo $24 - 9 = 15$
    possibilità.

    Ora contiamo in quanti modi possiamo riempire le singole cifre:
    \begin{itemize}
        \item la cifra dispari la possiamo riempire in 5 modi diversi;
        \item la prima cifra pari la possiamo riempire in 5 modi diversi (non è un problema se la prima cifra è uno 0);
        \item la seconda cifra pari la possiamo riempire in 4 modi diversi;
        \item la terza cifra pari la possiamo riempire in 3 modi diversi;
        \item consideriamo come quarta cifra quella ripetuta, per cui l'abbiamo già scelta.
    \end{itemize}

    Quindi, per ciascuna delle 15 possibilità calcolate prima, possiamo riempire le cifre in
    $5 \cdot 5 \cdot 4 \cdot 3 = 300$ modi diversi.

    Quindi abbiamo in totale $15 \cdot 300 = 4500$ combinazioni.
\end{soluzione}

\begin{soluzione}{ex:distrettuali_2008_7}
    Dobbiamo costruire un numero di 5 cifre divisibile per 11.
    Per il criterio di divisibilità la somma delle cifre di posizione dispari meno la somma delle cifre
    di posizione pari deve essere un multiplo di 11.
    Nel nostro numero avremo 3 cifre di posizioni dispari e 2 cifre di posizione pari.

    La somma delle cifre che abbiamo a disposizione è $1 + 2 + 4 + 7 + 9 = 23$.

    Non è un numero pari quindi non possiamo sperare di avere 11 come somma delle cifre di posizioni dispari e 11
    come somma delle cifre di posizione pari.

    Potremo tentare di suddividere questi numeri in modo che la differenza delle somme sia 22.
    Ma in questo caso avremo solo la combinazione $(2  + 4 + 7 + 9) - 1 = 22$, che non va bene perché abbiamo quattro
    cifre da una parte e una sola dall'altra.

    Non ci rimane che suddividere i numeri in modo da avere una differenza pari ad 11.
    Se chiamiamo $x$ la somma dei numeri di posizione dispari (o pari) e $y$ la somma dei numeri di posizione pari (o
    dispari), abbiamo:
    \begin{equation*}
        \begin{cases}
            x + y = 23 \\
            x - y = 11
        \end{cases}
    \end{equation*}

    La soluzione di questo sistema è $x = 17$ e $y = 6$.

    Il numero $x = 17$ non può essere la somma delle cifre di posizioni pari, perché gli unici due numeri minori di 10
    che sommati fanno 17 sono 8 e 9, ma non abbiamo l'8 tra le cifre utilizzabili.

    Quindi la somma delle cifre di posizioni dispari deve essere 17 e la somma delle cifre di posizioni pari deve essere
    6.

    L'unica combinazione per ottenere 6 come somma delle due cifre di posizioni pari è 2 e 4.
    Quindi per le cifre di posizione dispari possiamo scegliere tra 1, 7 e 9;
    per le cifre di posizioni pari possiamo scegliere tra 2 e 4.

    Per collocare le tre cifre di posizioni dispari mi basta calcolare le permutazioni di 3 elementi.
    Per collocare le due cifre di posizioni pari mi basta calcolare le permutazioni di 2 elementi.
    Quindi la quantità di numeri che possiamo comporre è pari a:
    \begin{equation*}
        P_3 \cdot P_2 = 3! \cdot 2! = 6 \cdot 2 = 12
    \end{equation*}
\end{soluzione}

\begin{soluzione}{ex:distrettuali_2016_5}
    \emph{Nota bene.} Lo sapevi che in un dado da sei facce la somma delle facce opposte deve fare 7?

    Cominciamo con il colorare la faccia con il numero 1: la possiamo colorare in 4 modi diversi.

    Ora passiamo alla faccia opposta, quella con il numero 6: la possiamo colorare in 3 modi diversi, perché il colore
    deve essere diverso da quello della faccia con il numero 1.

    Quindi, limitatamente alle facce 1 e 6, le possiamo colorare in $4 \cdot 3$ modi diversi.

    Lo stesso ragionamento lo possiamo fare considerando le coppie di facce 2 e 5, e 3 e 4.
    Quindi possiamo colorare il dado in $(4 \cdot 3)^3$ modi diversi.

    Ma questo non ci garantisce di usare almeno 3 colori diversi.
    Andiamo quindi a togliere le possibili colorazioni con solo 2 colori diversi.

    Per colorare il dado con solo due colori ho due possibilità per la faccia dell'1, due possibilità per la faccia del
    2, e due possibilità per la faccia del 3.
    Le altre facce vengono colorate di conseguenza.
    Quindi ho in totale $2^3$ possibilità.

    Però devo moltiplicare queste possibilità per il numero di possibili combinazioni di due colori scelti tra i quattro
    di partenza, che è pari alle combinazioni di quattro elementi in gruppi da 2.
    Quindi al totale devo sottrarre:
    \begin{equation*}
        2^3 \cdot \binom{4}{2} = 2^3 \cdot 3 \cdot 2
    \end{equation*}

    Ora, sfruttiamo qualche raccoglimento per semplificare il calcolo:
    \begin{gather*}
        (4 \cdot 3)^3 - 2^3 \cdot 3 \cdot 2 = \\
        = (2^2 \cdot 3)^3 - 2^4 \cdot 3 = \\
        = 2^6 \cdot 3^3 - 2^4 \cdot 3 = \\
        = 2^4 \cdot 3 \cdot (2^2 \cdot 3^2 - 1) = \\
        = 2^4 \cdot 3 \cdot 35 = \\
        = 2^4 \cdot 3 \cdot 5 \cdot 7
    \end{gather*}
\end{soluzione}